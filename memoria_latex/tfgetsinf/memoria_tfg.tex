%%%%%%%%%%%%%%%%%%%%%%%%%%%%%%%%%%%%%%%%%%%%%%%%%%%%%%%%%%%%%%%%%%%%%%%%%%%%%%%
%                       CARGA DE LA CLASE DE DOCUMENTO                        %
%%%%%%%%%%%%%%%%%%%%%%%%%%%%%%%%%%%%%%%%%%%%%%%%%%%%%%%%%%%%%%%%%%%%%%%%%%%%%%%

\documentclass[11pt,spanish,listoffigures,listoftables]{tfgetsinf}

%%%%%%%%%%%%%%%%%%%%%%%%%%%%%%%%%%%%%%%%%%%%%%%%%%%%%%%%%%%%%%%%%%%%%%%%%%%%%%%
%                          CODIFICACIÓN DEL FICHERO                           %
%%%%%%%%%%%%%%%%%%%%%%%%%%%%%%%%%%%%%%%%%%%%%%%%%%%%%%%%%%%%%%%%%%%%%%%%%%%%%%%

\usepackage[utf8]{inputenc} 
\usepackage{babel}
\usepackage{hyperref}
\usepackage{biblatex}

\addbibresource{bib.bib}  % Enlazar el archivo .bib
\addglobalbib{bib.bib}
\hypersetup{ colorlinks=true, linkcolor=black, urlcolor=cyan }

%%%%%%%%%%%%%%%%%%%%%%%%%%%%%%%%%%%%%%%%%%%%%%%%%%%%%%%%%%%%%%%%%%%%%%%%%%%%%%%
%                             OTROS PAQUETES                                  %
%%%%%%%%%%%%%%%%%%%%%%%%%%%%%%%%%%%%%%%%%%%%%%%%%%%%%%%%%%%%%%%%%%%%%%%%%%%%%%%
% (Aquí puedes añadir los paquetes que necesites)

%%%%%%%%%%%%%%%%%%%%%%%%%%%%%%%%%%%%%%%%%%%%%%%%%%%%%%%%%%%%%%%%%%%%%%%%%%%%%%%
%                         DATOS DEL TRABAJO                                   %
%%%%%%%%%%%%%%%%%%%%%%%%%%%%%%%%%%%%%%%%%%%%%%%%%%%%%%%%%%%%%%%%%%%%%%%%%%%%%%%

\title{Uso de IA en la detección de Artrosis para rodillas}
\author{Hernández Martínez, Carlos}
\tutor{Juan Ciscar, Alfonso}
\curs{2024-2025}

%%%%%%%%%%%%%%%%%%%%%%%%%%%%%%%%%%%%%%%%%%%%%%%%%%%%%%%%%%%%%%%%%%%%%%%%%%%%%%%
%            PALABRAS CLAVE Y RESÚMENES (en tres idiomas)                     %
%%%%%%%%%%%%%%%%%%%%%%%%%%%%%%%%%%%%%%%%%%%%%%%%%%%%%%%%%%%%%%%%%%%%%%%%%%%%%%%

\keywords{Palabras clave en catalán} % catalán
         {Palabras clave en español} % español
         {Keywords in English}       % inglés

\begin{document}

%%%%%%%%%%%%%%%%%%%%%%%%%%%%%%%%%%%%%%%%%%%%%%%%%%%%%%%%%%%%%%%%%%%%%%%%%%%%%%%
%                             RESÚMENES                                       %
%%%%%%%%%%%%%%%%%%%%%%%%%%%%%%%%%%%%%%%%%%%%%%%%%%%%%%%%%%%%%%%%%%%%%%%%%%%%%%%

\begin{abstract}
Aquí citamos a un datajkjkset \cite{gornale2020digital}.
Citación paper IEE \cite{10863523}, Dataset \cite{chen2018knee}
aqui citamos un paper \cite{VAATTOVAARA2025100580}
\end{abstract}

\begin{abstract}[spanish]
(Resumen en castellano)
\end{abstract}

\begin{abstract}[english]
(Resumen en inglés)
\end{abstract}

\mainmatter

%%%%%%%%%%%%%%%%%%%%%%%%%%%%%%%%%%%%%%%%%%%%%%%%%%%%%%%%%%%%%%%%%%%%%%%%%%%%%%%
%                              CAPÍTULO 1                                     %
%                                    INTRO                                     %
%%%%%%%%%%%%%%%%%%%%%%%%%%%%%%%%%%%%%%%%%%%%%%%%%%%%%%%%%%%%%%%%%%%%%%%%%%%%%%%

\chapter{Introducción}  % ~5 páginas

\section{Motivación}     % 1.1
La artritis es una de las enfermedades musculoesqueléticas más prevalentes a nivel mundial y una de las principales causas de discapacidad en adultos mayores. Su diagnóstico y seguimiento se basa tradicionalmente en la evaluación clínica y en la interpretación de imágenes médicas, como radiografías, resonancias magnéticas y tomografías computarizadas. Sin embargo, este proceso suele depender en gran medida de la experiencia del profesional médico, lo que puede generar variabilidad en los diagnósticos y retrasos en la detección temprana de la enfermedad.

En los últimos años, los avances en inteligencia artificial, y en particular en el aprendizaje profundo, han demostrado un gran potencial para mejorar la precisión y la eficiencia en el análisis de imágenes médicas. Las redes neuronales convolucionales (CNN) han sido ampliamente utilizadas en el campo de la visión por computadora para tareas como la detección de patologías en radiografías, la segmentación de tejidos en resonancias magnéticas y la clasificación de niveles de severidad en enfermedades degenerativas.

Este Trabajo de Fin de Grado (TFG) se motiva por la necesidad de desarrollar métodos automáticos y robustos para el análisis de la artritis mediante el uso de técnicas de aprendizaje profundo. En particular, se busca explorar el uso de redes neuronales para la clasificación de imágenes médicas, utilizando bases de datos estandarizadas como \textit{Mendeley dataset} \cite{chen2018knee}. La aplicación de estos modelos podría no solo optimizar el proceso de diagnóstico, sino también contribuir al desarrollo de herramientas de soporte a la decisión clínica, facilitando una intervención más temprana y personalizada para los pacientes.

La relevancia de este estudio radica en su potencial impacto en la práctica clínica. Un sistema basado en inteligencia artificial podría reducir la carga de trabajo de los especialistas, mejorar la objetividad del diagnóstico y ofrecer segundas opiniones automáticas que complementen la evaluación médica tradicional. Además, el desarrollo de estas tecnologías en el ámbito de la artritis podría sentar un precedente para su aplicación en otras enfermedades musculoesqueléticas, ampliando el alcance del aprendizaje profundo en el campo de la salud.

Además, se plantea la posibilidad de realizar \textit{transfer learning} utilizando modelos preentrenados en artritis humana para aplicarlos en el diagnóstico de artritis en gatos. Esta adaptación podría beneficiar la práctica veterinaria, proporcionando herramientas automatizadas para la evaluación de la enfermedad en animales y mejorando la precisión en su diagnóstico.

En este contexto, el presente trabajo busca contribuir al avance del uso de inteligencia artificial en la detección y análisis de la artritis, evaluando diferentes enfoques de redes neuronales y analizando su rendimiento en la clasificación de imágenes médicas. La motivación principal es demostrar la viabilidad y efectividad de estos modelos en un problema biomédico concreto, promoviendo la integración de tecnologías emergentes en el ámbito de la salud.


\section{Objetivos}      % 1.2
% Presenta aquí los 3 objetivos principales
% p.e. Objetivo 1, Objetivo 2, Objetivo 3

\section{Estructura del documento}  % 1.3
% Describe la organización de los capítulos y secciones

%%%%%%%%%%%%%%%%%%%%%%%%%%%%%%%%%%%%%%%%%%%%%%%%%%%%%%%%%%%%%%%%%%%%%%%%%%%%%%%
%                              CAPÍTULO 2                                     %
%                                PRELIMINARES                                 %
%%%%%%%%%%%%%%%%%%%%%%%%%%%%%%%%%%%%%%%%%%%%%%%%%%%%%%%%%%%%%%%%%%%%%%%%%%%%%%%

\chapter{Preliminares}  % ~15 páginas

\section{Aprendizaje automático}

El aprendizaje automático (\textit{Machine Learning, ML}) es una rama de la inteligencia artificial que ha transformado múltiples disciplinas al permitir que los sistemas aprendan y mejoren su desempeño en tareas específicas a partir de la experiencia, sin necesidad de ser programados explícitamente para cada situación. Este enfoque se basa en la construcción de modelos matemáticos capaces de identificar patrones y relaciones en grandes volúmenes de datos, lo que resulta esencial para aplicaciones tan diversas como la visión por computadora, el procesamiento del lenguaje natural, la bioinformática y, en particular, el diagnóstico médico.

En el contexto del análisis de imágenes médicas, el aprendizaje automático permite automatizar procesos de detección, clasificación y segmentación de patologías, facilitando diagnósticos más rápidos y precisos. Por ejemplo, modelos basados en redes neuronales profundas han alcanzado desempeños comparables a los de expertos humanos en la identificación de anomalías en radiografías, resonancias magnéticas y tomografías computarizadas.

\subsection{Categorías de Aprendizaje Automático}

El aprendizaje automático se clasifica en tres paradigmas fundamentales:

\begin{itemize} \item \textbf{Aprendizaje supervisado}: Se apoya en conjuntos de datos etiquetados, donde cada entrada cuenta con una salida o etiqueta conocida. Durante el entrenamiento, el modelo aprende a mapear las entradas a las salidas correctas, facilitando la predicción de etiquetas en datos nuevos. Entre los algoritmos más comunes en este paradigma se encuentran la regresión lineal, las máquinas de soporte vectorial (\textit{Support Vector Machines, SVM}) y las redes neuronales profundas.

java
Copy
\item \textbf{Aprendizaje no supervisado}: Aquí, los datos carecen de etiquetas y el objetivo es descubrir patrones o estructuras subyacentes. Técnicas como el clustering (agrupamiento), la reducción de dimensionalidad mediante análisis de componentes principales (PCA) y los modelos generativos permiten identificar relaciones ocultas en los datos y organizar la información de forma significativa.

\item \textbf{Aprendizaje por refuerzo}: En este enfoque, un agente interactúa con un entorno y aprende a tomar decisiones mediante un sistema de prueba y error, optimizando sus acciones en función de una función de recompensa. Este paradigma es especialmente útil en problemas de toma de decisiones secuenciales y en la optimización de estrategias en entornos dinámicos.
\end{itemize}

\subsection{Procesamiento de Datos en Aprendizaje Automático}

El rendimiento y la capacidad de generalización de los modelos dependen en gran medida de la calidad de los datos de entrada. Por ello, el procesamiento de datos es una etapa crítica que involucra varias fases:

\begin{itemize} \item \textbf{Preprocesamiento}: Incluye la limpieza, normalización y transformación de los datos para eliminar ruidos, gestionar valores faltantes y unificar las escalas de las variables. En el ámbito de imágenes médicas, esta fase puede implicar la corrección de artefactos, la estandarización de intensidades y la segmentación preliminar de regiones de interés. \item \textbf{División del conjunto de datos}: Se segmenta la información en subconjuntos de entrenamiento, validación y prueba. Esta división es crucial para ajustar los hiperparámetros del modelo, prevenir el sobreajuste y evaluar de manera objetiva el desempeño final. \item \textbf{Extracción y selección de características}: En ciertos casos, se realiza una selección o generación de características relevantes que potencien la capacidad del modelo para identificar patrones significativos, lo cual es particularmente importante en dominios con alta dimensionalidad o datos complejos. \end{itemize}

\subsection{Métricas de Evaluación}

Para medir la efectividad de un modelo de aprendizaje automático, se emplean diversas métricas que varían según el tipo de problema:

\begin{itemize} \item \textbf{Para clasificación}: Se utilizan métricas como la precisión (\textit{accuracy}), la precisión (precision), el recall (sensibilidad) y el F1-score, especialmente en contextos con clases desbalanceadas. La curva ROC y el área bajo la curva (AUC-ROC) son también indicadores fundamentales para evaluar la capacidad del modelo de distinguir entre clases. \item \textbf{Para regresión}: Se valora el desempeño mediante el error cuadrático medio (MSE) y el error absoluto medio (MAE), que cuantifican la diferencia entre las predicciones del modelo y los valores reales. \end{itemize}

\subsection{Aplicaciones del Aprendizaje Automático en Imágenes Médicas}

La aplicación del aprendizaje automático en el análisis de imágenes médicas ha abierto nuevas fronteras en el diagnóstico y seguimiento de diversas patologías. Algunas de las aplicaciones más destacadas incluyen:

\begin{itemize} \item \textbf{Detección de enfermedades}: Mediante modelos supervisados, se pueden identificar anomalías en radiografías, resonancias magnéticas y tomografías computarizadas, facilitando diagnósticos tempranos y precisos. \item \textbf{Segmentación de tejidos y órganos}: Algoritmos basados en redes neuronales permiten delimitar estructuras anatómicas y patologías en imágenes médicas, lo que es fundamental para planificar tratamientos y cirugías. \item \textbf{Clasificación de tumores}: Los modelos de aprendizaje profundo pueden diferenciar entre tumores benignos y malignos, aportando una segunda opinión automatizada que complementa la evaluación clínica. \end{itemize}

En este trabajo se explora la aplicación del aprendizaje profundo para la detección de artritis en imágenes de rodillas. La integración de técnicas avanzadas de ML en el análisis de imágenes médicas no solo mejora la precisión diagnóstica, sino que también reduce la variabilidad interobservador, ofreciendo un soporte robusto y automatizado que puede transformar el proceso de diagnóstico en el ámbito clínico. Con el continuo avance en el procesamiento de datos y el desarrollo de modelos más sofisticados, se espera que el aprendizaje automático siga abriendo nuevas posibilidades para la personalización y optimización del tratamiento médico.








\section{Redes neuronales}           % 2.2
% Descripción breve de redes neuronales

\section{ML aplicado a CV y tareas biomédicas (MedMNIST)} % 2.3
% FIXME: Detalla aquí cómo se aplica ML en Computer Vision 
% y en particular a tareas médicas con la herramienta MedMNIST

%%%%%%%%%%%%%%%%%%%%%%%%%%%%%%%%%%%%%%%%%%%%%%%%%%%%%%%%%%%%%%%%%%%%%%%%%%%%%%%
%                              CAPÍTULO 3                                     %
%                      PRIMERA CONTRIBUCIÓN (OBJETIVO 1)                      %
%%%%%%%%%%%%%%%%%%%%%%%%%%%%%%%%%%%%%%%%%%%%%%%%%%%%%%%%%%%%%%%%%%%%%%%%%%%%%%%

\chapter{Capítulo 1 de contribución}  % ~15 páginas

Replicar el trabajo de \cite{10863523} utilizando este datset \cite{chen2018knee}

El primer objetivo de este trabajo es replicar el estudio realizado en \cite{10863523} utilizando el conjunto de datos proporcionado en \cite{chen2018knee}. Para ello, se emplea el modelo **EfficientNetB5** preentrenado, con modificaciones en la capa de clasificación para adaptarlo al problema específico. La nueva estructura de la capa de clasificación es la siguiente:

\begin{verbatim}
self.efficientnet.classifier = nn.Sequential(
    nn.Linear(in_features, 256),
    nn.ReLU(),
    nn.Linear(256, 256),
    nn.ReLU(),
    nn.Linear(256, num_classes),
    nn.Softmax(dim=1)
)
\end{verbatim}

Para el entrenamiento del modelo, se emplea la función de pérdida **CrossEntropy**, complementada con regularización **L1** y **L2** con un coeficiente de penalización de 0.001. Además, se utilizan los siguientes hiperparámetros:

\begin{itemize}
    \item \textbf{Paciencia} (*patience*): 5
    \item \textbf{Factor de reducción del *learning rate*} (*factor*): 0.1
    \item \textbf{Tasa de aprendizaje inicial} (*lr*): 0.001
    \item \textbf{Betas}: (0.9, 0.999)
\end{itemize}

El proceso de optimización se lleva a cabo con el optimizador **Adam**, mientras que el ajuste dinámico de la tasa de aprendizaje se realiza mediante **ReduceLROnPlateau**:

\begin{verbatim}
optimizer = torch.optim.Adam(self.model.parameters(),
                             lr=self.learning_rate,
                             betas=self.betas)

scheduler = torch.optim.lr_scheduler.ReduceLROnPlateau(optimizer=optimizer,
                                                       factor=self.factor,
                                                       patience=self.patience)
\end{verbatim}

Este diseño garantiza un entrenamiento eficiente y adaptado al conjunto de datos empleado, permitiendo evaluar la reproducibilidad de los resultados obtenidos en el estudio original.

Este diseño garantiza un entrenamiento eficiente y adaptado al conjunto de datos empleado, permitiendo evaluar la reproducibilidad de los resultados obtenidos en el estudio original.



Primero repetir el experimento sin hacer data augmentation,

para el modelo 
\chapter{Capítulo 2 de contribución}   % ~15 páginas
% Expón aquí tu segundo objetivo y resultados derivados

%%%%%%%%%%%%%%%%%%%%%%%%%%%%%%%%%%%%%%%%%%%%%%%%%%%%%%%%%%%%%%%%%%%%%%%%%%%%%%%
%                              CAPÍTULO 5                                     %
%                     TERCERA CONTRIBUCIÓN (OBJETIVO 3)                       %
%%%%%%%%%%%%%%%%%%%%%%%%%%%%%%%%%%%%%%%%%%%%%%%%%%%%%%%%%%%%%%%%%%%%%%%%%%%%%%%

\chapter{Capítulo 3 de contribución}   % ~15 páginas
% Expón aquí tu tercer objetivo y resultados derivados

%%%%%%%%%%%%%%%%%%%%%%%%%%%%%%%%%%%%%%%%%%%%%%%%%%%%%%%%%%%%%%%%%%%%%%%%%%%%%%%
%                              CAPÍTULO 6                                     %
%                                CONCLUSIONES                                 %
%%%%%%%%%%%%%%%%%%%%%%%%%%%%%%%%%%%%%%%%%%%%%%%%%%%%%%%%%%%%%%%%%%%%%%%%%%%%%%%

\chapter{Conclusiones}  % ~5 páginas

\section{Resumen del trabajo realizado} % 6.1
% Repasa y sintetiza las secciones principales

\section{Objetivos alcanzados}         % 6.2
% Verifica si se cumplieron los objetivos planteados

\section{Trabajo futuro}               % 6.3
% Explica las posibles extensiones o mejoras

%%%%%%%%%%%%%%%%%%%%%%%%%%%%%%%%%%%%%%%%%%%%%%%%%%%%%%%%%%%%%%%%%%%%%%%%%%%%%%%
%                              BIBLIOGRAFÍA                                   %
%%%%%%%%%%%%%%%%%%%%%%%%%%%%%%%%%%%%%%%%%%%%%%%%%%%%%%%%%%%%%%%%%%%%%%%%%%%%%%%

\printbibliography 
\cleardoublepage

%%%%%%%%%%%%%%%%%%%%%%%%%%%%%%%%%%%%%%%%%%%%%%%%%%%%%%%%%%%%%%%%%%%%%%%%%%%%%%%
%                           APÉNDICES (OPCIONALES)                            %
%%%%%%%%%%%%%%%%%%%%%%%%%%%%%%%%%%%%%%%%%%%%%%%%%%%%%%%%%%%%%%%%%%%%%%%%%%%%%%%

\APPENDIX

\chapter{Configuración del sistema}
% ...

\chapter{Otro apéndice}
% ...

\end{document}
